% Options for packages loaded elsewhere
\PassOptionsToPackage{unicode}{hyperref}
\PassOptionsToPackage{hyphens}{url}
%
\documentclass[
  12pt,
  ignorenonframetext,
  progressbar=frametitle]{beamer}
\usepackage{pgfpages}
\setbeamertemplate{caption}[numbered]
\setbeamertemplate{caption label separator}{: }
\setbeamercolor{caption name}{fg=normal text.fg}
\beamertemplatenavigationsymbolsempty
% Prevent slide breaks in the middle of a paragraph
\widowpenalties 1 10000
\raggedbottom
\setbeamertemplate{part page}{
  \centering
  \begin{beamercolorbox}[sep=16pt,center]{part title}
    \usebeamerfont{part title}\insertpart\par
  \end{beamercolorbox}
}
\setbeamertemplate{section page}{
  \centering
  \begin{beamercolorbox}[sep=12pt,center]{part title}
    \usebeamerfont{section title}\insertsection\par
  \end{beamercolorbox}
}
\setbeamertemplate{subsection page}{
  \centering
  \begin{beamercolorbox}[sep=8pt,center]{part title}
    \usebeamerfont{subsection title}\insertsubsection\par
  \end{beamercolorbox}
}
\AtBeginPart{
  \frame{\partpage}
}
\AtBeginSection{
  \ifbibliography
  \else
    \frame{\sectionpage}
  \fi
}
\AtBeginSubsection{
  \frame{\subsectionpage}
}
\usepackage{lmodern}
\usepackage{amssymb,amsmath}
\usepackage{ifxetex,ifluatex}
\ifnum 0\ifxetex 1\fi\ifluatex 1\fi=0 % if pdftex
  \usepackage[T1]{fontenc}
  \usepackage[utf8]{inputenc}
  \usepackage{textcomp} % provide euro and other symbols
\else % if luatex or xetex
  \usepackage{unicode-math}
  \defaultfontfeatures{Scale=MatchLowercase}
  \defaultfontfeatures[\rmfamily]{Ligatures=TeX,Scale=1}
\fi
\usetheme[]{metropolis}
% Use upquote if available, for straight quotes in verbatim environments
\IfFileExists{upquote.sty}{\usepackage{upquote}}{}
\IfFileExists{microtype.sty}{% use microtype if available
  \usepackage[]{microtype}
  \UseMicrotypeSet[protrusion]{basicmath} % disable protrusion for tt fonts
}{}
\makeatletter
\@ifundefined{KOMAClassName}{% if non-KOMA class
  \IfFileExists{parskip.sty}{%
    \usepackage{parskip}
  }{% else
    \setlength{\parindent}{0pt}
    \setlength{\parskip}{6pt plus 2pt minus 1pt}}
}{% if KOMA class
  \KOMAoptions{parskip=half}}
\makeatother
\usepackage{xcolor}
\IfFileExists{xurl.sty}{\usepackage{xurl}}{} % add URL line breaks if available
\IfFileExists{bookmark.sty}{\usepackage{bookmark}}{\usepackage{hyperref}}
\hypersetup{
  pdftitle={International Organisations},
  pdfauthor={Susmit Islam},
  hidelinks,
  pdfcreator={LaTeX via pandoc}}
\urlstyle{same} % disable monospaced font for URLs
\newif\ifbibliography
\setlength{\emergencystretch}{3em} % prevent overfull lines
\providecommand{\tightlist}{%
  \setlength{\itemsep}{0pt}\setlength{\parskip}{0pt}}
\setcounter{secnumdepth}{-\maxdimen} % remove section numbering
\setbeamersize{text margin left=10mm,text margin right=10mm}
\setbeamertemplate{frametitle continuation}{}
\usepackage{bangla}
\usepackage{enumitem}
\setlistdepth{20}
\renewlist{itemize}{itemize}{20}
\renewlist{enumerate}{enumerate}{20}
\setlist[itemize]{label=$*$}
\setlist[itemize,1]{label=\tiny$\blacksquare$}
\setlist[itemize,2]{label=$\bullet$}
\setlist[itemize,3]{label=$\boldsymbol\circ$}
\setlist[itemize,4]{label=$*$}
\linespread{1.3}

\title{International Organisations}
\subtitle{The Bare Minimum for BCS}
\author{Susmit Islam}
\date{}
\institute{Sir Salimullah Medical College}

\begin{document}
\frame{\titlepage}
\begin{frame}[t,allowframebreaks]
{Table of contents}\setbeamertemplate{section in toc}[sections numbered]

\tableofcontents[hideallsubsections]
\end{frame}

\section{Preface}
\begin{frame}[allowframebreaks]
{Preface}
\protect\hypertarget{preface}{}
What follows are my notes on International Organisations. Many of these
things change quite frequently and so these notes may be out of date
when you read them. Proceed with caution. There's probably plenty of
mistakes, all my own. Inform me whenever you find one.\newline

\textbf{Use at your own peril.}

\hfill Susmit Islam

\hfill\today
\end{frame}

\section{Commonwealth of Nations}
\begin{frame}[allowframebreaks]
{Commonwealth of Nations}
\protect\hypertarget{commonwealth-of-nations}{}
\begin{itemize}
\tightlist
\item
  \textbf{Intro:} political organisation, largely composed of former
  British colonies
\item
  \textbf{Establishment:}

  \begin{itemize}
  \tightlist
  \item
    \textbf{Balfour declaration:} \(\rightarrow\) ``British Commonwealth
    of Nations'' (1926)
  \item
    Statute of Westminster (1931) made the dominions their own sovereign
    states
  \item
    \textbf{London declaration} \(\rightarrow\) ``Commonwealth of
    Nations'' (1949)
  \end{itemize}
\item
  \textbf{Members:} 56

  \begin{itemize}
  \tightlist
  \item
    \textbf{Latest members:} Togo, Gabon (2022)
  \item
    The last 4 members are the only ones that have never been British
    colonies

    \begin{itemize}
    \tightlist
    \item
      Mozambique (1st non-colony member; 2009)
    \item
      Rwanda (2009)
    \item
      Togo (2022)
    \item
      Gabon (2022)
    \end{itemize}
  \end{itemize}
\item
  \textbf{HQ:} Marlborough house, London.
\item
  \textbf{Head:} Queen Elizabeth II.
\end{itemize}
\end{frame}

\section{NAM (Non-Aligned Movement)}
\begin{frame}[allowframebreaks]
{NAM (Non-Aligned Movement)}
\protect\hypertarget{nam-non-aligned-movement}{}
\begin{itemize}
\tightlist
\item
  \textbf{Intro:} Formed during the Cold War by countries supporting
  neither the Soviet nor the US.
\item
  \textbf{Establishment:} 1961; Belgrade, Yugoslavia.
\item
  \textbf{Members:} 120 members, 20 observers.
\item
  \textbf{HQ:} None
\end{itemize}
\end{frame}

\section{OIC (Organisation of Islamic Co-operation)}
\begin{frame}[allowframebreaks]
{OIC (Organisation of Islamic Co-operation)}
\protect\hypertarget{oic-organisation-of-islamic-co-operation}{}
\begin{itemize}
\tightlist
\item
  \textbf{Intro:} Made in the wake of the Arab-Israel war
\item
  \textbf{Establishment:}

  \begin{itemize}
  \tightlist
  \item
    Sep 25, 1969, as Organisation of Islamic Conference
  \item
    Changed to current name in 2011
  \end{itemize}
\item
  \textbf{Members:} 57
\item
  \textbf{HQ:} Jeddah
\item
  \textbf{1st sec.~gen.:} Tunku Abdul Rahman (Malaysia)
\item
  \textbf{Current sec.~gen.:} Hissein Brahim Taha (Chad)
\end{itemize}
\end{frame}

\section{EU (European Union)}
\begin{frame}[allowframebreaks]
{EU (European Union)}
\protect\hypertarget{eu-european-union}{}
\begin{itemize}
\tightlist
\item
  \textbf{Intro:}

  \begin{itemize}
  \tightlist
  \item
    World's largest economic union
  \item
    Political and economic union, with an internal single market
  \item
    After the end of the Cold War and the unification of Germany, to
    ``further advance European integration''
  \end{itemize}
\item
  \textbf{Establishment:}

  \begin{itemize}
  \tightlist
  \item
    \textbf{Treaty of Rome:} (1957)

    \begin{itemize}
    \tightlist
    \item
      Established \textbf{European Economic Community (EEC)}
    \item
      Signed by 6 countries: Belgium, France, Netherlands, W. Germany,
      Italy, Luxembourg
    \item
      Signed: Mar 25, 1957
    \item
      Effective: Jan 1, 1958
    \item
      After formation of EU, renamed to European Community (EC)
    \item
      Was one of the ``3 pillars'' of EU till 2009, when the 3-pillar
      system was abolished and the functions of the pillars were
      absorbed into EU
    \end{itemize}
  \item
    \textbf{Treaty of Maastricht:} (1992; Maastricht, Netherlands)

    \begin{itemize}
    \tightlist
    \item
      Signed by 12 countries marking the beginning of EU
    \item
      Signed: Feb 7, 1992.
    \item
      Effective: Nov 1, 1993.
    \end{itemize}
  \item
    \textbf{Treaty of Lisbon:} (2007)

    \begin{itemize}
    \tightlist
    \item
      Initially called ``Reform Treaty''
    \item
      Amendment of the Maastricht treaty
    \item
      Signed: Dec 13, 2007
    \item
      Effective: Dec 1, 2009
    \item
      Abolished the 3-pillar system and marked the emergence of the EU
      as a legal person
    \item
      In 2012, the EU received the Nobel Peace Prize
    \end{itemize}
  \end{itemize}
\item
  \textbf{Members:} 27
\item
  \textbf{HQ:} Brussels, Belgium.
\item
  \textbf{Structure:}

  \begin{itemize}
  \tightlist
  \item
    7 institutions, 1st 4 are the main decision-making institutions

    \begin{itemize}
    \item
      \fbox{European Parliament}

      \begin{itemize}
      \tightlist
      \item
        Location: Strasbourg (France), Brussels, Luxembourg

        \begin{itemize}
        \tightlist
        \item
          Members: 705 MEPs (members of European parliament)
        \end{itemize}
      \end{itemize}
    \item
      \fbox{European Council}

      \begin{itemize}
      \tightlist
      \item
        Location: Brussels

        \begin{itemize}
        \tightlist
        \item
          Members: Heads of state of EU member countries
        \item
          President: Charles Michel
        \end{itemize}
      \end{itemize}
    \item
      \fbox{Council of the EU}

      \begin{itemize}
      \tightlist
      \item
        Location: Brussels

        \begin{itemize}
        \tightlist
        \item
          Members: Ministers of EU member countries
        \end{itemize}
      \end{itemize}
    \item
      \fbox{European Commission}
    \item
      \fbox{Court of Justice of the EU}

      \begin{itemize}
      \tightlist
      \item
        Location: Luxembourg
      \end{itemize}
    \item
      \fbox{European Central Bank}

      \begin{itemize}
      \tightlist
      \item
        Location: Frankfurt, Germany

        \begin{itemize}
        \tightlist
        \item
          Members: ECB President and Vice-President and governors of
          national central banks from all EU countries
        \end{itemize}
      \end{itemize}
    \end{itemize}
  \end{itemize}
\item
  \textbf{Euro:}

  \begin{itemize}
  \tightlist
  \item
    ``Members'':

    \begin{itemize}
    \tightlist
    \item
      19 member states, together called \textbf{eurozone / euro area},
      have adopted the euro as their primary currency.
    \item
      Croatia will become the 20th member on Jan 1, 2023
    \item
      Other EU members can join the eurozone when they fulfill the
      criteria for joining
    \end{itemize}
  \item
    Launch:

    \begin{itemize}
    \tightlist
    \item
      Jan 1, 1999
    \item
      11 member states had met the criteria back then, and hence those
      11 were the firsts to adopt the euro
    \item
      Physical notes and coins were launched on Jan 1, 2002.
    \end{itemize}
  \end{itemize}
\end{itemize}
\end{frame}

\section{AU (African Union)}
\begin{frame}[allowframebreaks]
{AU (African Union)}
\protect\hypertarget{au-african-union}{}
\begin{itemize}
\tightlist
\item
  \textbf{Establishment:}

  \begin{itemize}
  \tightlist
  \item
    \textbf{OAU (Organisation of African Unity)} \(\rightarrow\) May 25,
    1963
  \item
    \textbf{Sirte declaration} \(\rightarrow\) Sep 9, 1999 (9/9/99);
    signed by heads of OAU states
  \item
    \textbf{AU} officially launched in Jul 9, 2002.
  \end{itemize}
\item
  \textbf{Members:} 55

  \begin{itemize}
  \tightlist
  \item
    \textbf{Latest member:} Morocco (2017)
  \item
    All African nations are members
  \item
    Morocco was a founding member of OAU, but quit in 1984 due to the
    acceptance of Sahrawi Arab Democratic Republic as a member state,
    most of the territory of which is ruled by Morocco. Nonetheless, it
    is a partially recognised sovereign state, with 41 UN nations
    recognising it.
  \end{itemize}
\item
  \textbf{HQ:} Addis Ababa.
\item
  \textbf{Chairperson:} Macky Sall (President of Senegal)
\end{itemize}
\end{frame}

\section{Arab League}
\begin{frame}[allowframebreaks]
{Arab League}
\protect\hypertarget{arab-league}{}
\begin{itemize}
\tightlist
\item
  \textbf{Establishment:}

  \begin{itemize}
  \tightlist
  \item
    \textbf{Alexandria protocol} (1945) \(\rightarrow\) \textbf{Arab
    league} (Mar 22, 1945)
  \end{itemize}
\item
  \textbf{Members:}

  \begin{itemize}
  \tightlist
  \item
    Member states: 22
  \item
    Observer states: 7

    \begin{itemize}
    \tightlist
    \item
      Armenia
    \item
      Brazil
    \item
      Chad
    \item
      Eritrea
    \item
      Greece
    \item
      India
    \item
      Venezuela
    \end{itemize}
  \item
    Latest member: Comoros (1993)
  \item
    Suspended member: Syria (since 2011)
  \end{itemize}
\item
  \textbf{Probably unrelated fact:} Egypt and Syria formed the
  \textbf{United Arab Republic} in 1958 (capital: Cairo). In 1961, Syria
  quit the union, and Egypt continued to be officially known as the UAR
  until 1971.
\end{itemize}
\end{frame}

\section{OAS (Organization of American States)}
\begin{frame}[allowframebreaks]
{OAS (Organization of American States)}
\protect\hypertarget{oas-organization-of-american-states}{}
\begin{itemize}
\tightlist
\item
  \textbf{Establishment:} Apr 30, 1948.
\item
  \textbf{Members:} 35

  \begin{itemize}
  \tightlist
  \item
    All independent nations of the Americas are members
  \item
    Founding members: 21
  \item
    Latest member: Guayana (1991)
  \end{itemize}
\end{itemize}
\end{frame}

\section{GCC (Gulf Cooperation Council)}
\begin{frame}[allowframebreaks]
{GCC (Gulf Cooperation Council)}
\protect\hypertarget{gcc-gulf-cooperation-council}{}
\begin{itemize}
\tightlist
\item
  \textbf{Intro:} Union of countries bordering the Persian gulf
\item
  \textbf{Establishment:} May 25, 1981
\item
  \textbf{Members:} Bahrain, Kuwait, Oman, Qatar, KSA, UAE
\item
  \textbf{HQ:} Riyadh
\end{itemize}
\end{frame}

\section{G-7 (Group of 7)}
\begin{frame}[allowframebreaks]
{G-7 (Group of 7)}
\protect\hypertarget{g-7-group-of-7}{}
\begin{itemize}
\tightlist
\item
  \textbf{Intro:} Avengers, basically. Big-shot governments join forces
  to discuss and coordinate solutions to major global issues in trade,
  security, economics, and climate change.
\item
  \textbf{Members:} 7

  \begin{itemize}
  \tightlist
  \item
    USA
  \item
    Canada
  \item
    UK
  \item
    Germany
  \item
    France
  \item
    Italy
  \item
    Japan
  \end{itemize}
\item
  \textbf{Establishment:}

  \begin{itemize}
  \tightlist
  \item
    Mar 25, 1973 \(\rightarrow\) G5
  \item
    Nov 15, 1975 \(\rightarrow\) G6 (Italy added)
  \item
    1976 \(\rightarrow\) G7 (Canada added)
  \item
    1997 \(\rightarrow\) G8 (Russia added)
  \item
    2014 \(\rightarrow\) G7 (Russia suspended over annexation of Crimea)
  \end{itemize}
\item
  \textbf{HQ:} None
\end{itemize}
\end{frame}

\section{G-20}
\begin{frame}[allowframebreaks]
{G-20}
\protect\hypertarget{g-20}{}
\begin{itemize}
\tightlist
\item
  \textbf{Intro:} ``the primary venue for international economic and
  financial cooperation''
\item
  \textbf{Establishment:} Sep 26, 1999.
\item
  \textbf{Members:} 20

  \begin{itemize}
  \tightlist
  \item
    19 countries + the EU

    \begin{itemize}
    \tightlist
    \item
      G-7 countries (7)
    \item
      BRICS countries (5): Brazil, Russia, India, China, South Africa
    \item
      Others (7): ARG, AUS, KSA, KOR (Republic of Korea aka South
      Korea), MEX, IDN (Indonesia), TUR (Turkey)
    \end{itemize}
  \end{itemize}
\item
  \textbf{HQ:} None
\end{itemize}
\end{frame}

\section{D-8 (Developing 8)}
\begin{frame}[allowframebreaks]
{D-8 (Developing 8)}
\protect\hypertarget{d-8-developing-8}{}
\begin{itemize}
\tightlist
\item
  \textbf{Establishment:}

  \begin{itemize}
  \tightlist
  \item
    \textbf{Istanbul declaration} (Jun 15, 1997) \(\rightarrow\) D-8
  \end{itemize}
\item
  \textbf{Members:} 8 states. Mnemonic: \banglatext{বাপ মা নাই তুমিই সব}

  \begin{itemize}
  \tightlist
  \item
    Bangladesh
  \item
    Pakistan
  \item
    Malaysia
  \item
    Nigeria
  \item
    Turkey
  \item
    Egypt (\banglatext{মিশর})
  \item
    Iran
  \end{itemize}
\item
  \textbf{HQ:} Istanbul
\end{itemize}
\end{frame}

\section{OPEC (Organisation of Petroleum Exporting Countries)}
\begin{frame}[allowframebreaks]
{OPEC (Organisation of Petroleum Exporting Countries)}
\protect\hypertarget{opec-organisation-of-petroleum-exporting-countries}{}
\begin{itemize}
\tightlist
\item
  \textbf{Intro:} Control oil prices worldwide. Responsible for 44\% of
  the total oil production worldwide.
\item
  \textbf{Establishment:}

  \begin{itemize}
  \tightlist
  \item
    Through Baghdad conference (Sep 10-14, 1960), attended by the
    founding 5

    \begin{itemize}
    \tightlist
    \item
      Iraq
    \item
      Iran
    \item
      KSA
    \item
      Kuwait
    \item
      Venezuela
    \end{itemize}
  \end{itemize}
\item
  \textbf{Members:} 13

  \begin{itemize}
  \tightlist
  \item
    Founding 5
  \item
    Other current members:

    \begin{itemize}
    \tightlist
    \item
      Algeria
    \item
      Angola
    \item
      Congo
    \item
      Equatorial Guinea
    \item
      Gabon
    \item
      Libya
    \item
      Nigeria
    \item
      UAE
    \end{itemize}
  \item
    Former members: Ecuador, Indonesia, Qatar
  \end{itemize}
\item
  \textbf{HQ:} Vienna
\end{itemize}
\end{frame}

\section{APEC (Asia-Pacific Economic Co-operation)}
\begin{frame}[allowframebreaks]
{APEC (Asia-Pacific Economic Co-operation)}
\protect\hypertarget{apec-asia-pacific-economic-co-operation}{}
\begin{itemize}
\tightlist
\item
  \textbf{Intro:} Promotes free trade throughout countries in the
  pacific rim
\item
  \textbf{Establishment:} 1989
\item
  \textbf{Members:} 21
\item
  \textbf{HQ:} Singapore
\end{itemize}
\end{frame}

\section{CIRDAP}
\begin{frame}[allowframebreaks]
{CIRDAP}
\protect\hypertarget{cirdap}{}
\begin{itemize}
\tightlist
\item
  Centre on Integrated Rural Development for Asia and the Pacific
\item
  \textbf{Intro:} \emph{Bangladesh based} organisation for rural
  development and poverty alleviation
\item
  \textbf{Establishment:}

  \begin{itemize}
  \tightlist
  \item
    Date: Jul 6, 1979
  \item
    Initiative by countries in the Asia-Pacific region and FAO of the UN
  \end{itemize}
\item
  \textbf{Members:}

  \begin{itemize}
  \tightlist
  \item
    Founding 6
  \item
    Currently 15
  \end{itemize}
\item
  \textbf{HQ:} Chameli house, Dhaka (in front of the Supreme Court)
\item
  \textbf{Director General:} Dr Cherdsak Virapat (Thailand)
\end{itemize}
\end{frame}

\section{ASEAN}
\begin{frame}[allowframebreaks]
{ASEAN}
\protect\hypertarget{asean}{}
\begin{itemize}
\tightlist
\item
  Association of South East Asian Nations
\item
  \textbf{Establishment:}

  \begin{itemize}
  \tightlist
  \item
    Bangkok declaration / ASEAN declaration:

    \begin{itemize}
    \tightlist
    \item
      Aug 8, 1967
    \item
      Signed by founding 5: Indonesia, Malaysia, Philippines, Singapore,
      Thailand
    \end{itemize}
  \end{itemize}
\item
  \textbf{Members:} 10 countries. Mnemonic: MTV \banglatext{তে} FILM
  \banglatext{দেখলে} BCS \banglatext{হবেনা}

  \begin{itemize}
  \tightlist
  \item
    Malaysia
  \item
    Thailand
  \item
    Vietnam
  \item
    Philippines
  \item
    Indonesia
  \item
    Laos
  \item
    Myanmar
  \item
    Brunei
  \item
    Cambodia
  \item
    Singapore
  \end{itemize}
\item
  \textbf{HQ:} Jakarta
\item
  \textbf{Summit:} Every 3 yrs.
\end{itemize}
\end{frame}

\section{SAARC}
\begin{frame}[allowframebreaks]
{SAARC}
\protect\hypertarget{saarc}{}
\begin{itemize}
\tightlist
\item
  South Asian Association for Regional Cooperation
\item
  \textbf{Establishment:}

  \begin{itemize}
  \tightlist
  \item
    Dec 8, 1985: signing of SAARC declaration
  \end{itemize}
\item
  \textbf{Members:} 8 members, 9 observers.

  \begin{itemize}
  \tightlist
  \item
    Members mnemonic: BNP IS MBA.

    \begin{itemize}
    \tightlist
    \item
      Bangladesh
    \item
      Nepal
    \item
      Pakistan
    \item
      India
    \item
      Sri Lanka
    \item
      Maldives
    \item
      Bhutan
    \item
      Afghanistan
    \end{itemize}
  \item
    Latest member: AFG (2007)
  \end{itemize}
\item
  \textbf{HQ:} Kathmandu
\item
  \textbf{Summit:}

  \begin{itemize}
  \tightlist
  \item
    No specific interval, although the official charter requires a
    summit every 18 mo
  \item
    First summit: Dec 7-8, 1985
  \item
    Latest (20th): 2022 (to be held)
  \end{itemize}
\item
  \textbf{Sec. Gen.:}

  \begin{itemize}
  \tightlist
  \item
    1st: Abul Ahsan (BD)
  \item
    Current: Esala Ruwan Weerakoon (SL)
  \end{itemize}
\end{itemize}
\end{frame}

\section{BIMSTEC}
\begin{frame}[allowframebreaks]
{BIMSTEC}
\protect\hypertarget{bimstec}{}
\begin{itemize}
\tightlist
\item
  Bay of Bengal Initiative for Multi-Sectorial Technical and Economic
  Cooperation
\item
  \textbf{Establishment:}

  \begin{itemize}
  \tightlist
  \item
    Jun 6, 1997: started as ``BIST-EC'' (BD, IND, SL, Thailand Economic
    Coop)
  \item
    Dec 22, 1997: Myanmar joined, renamed to ``BIMST-EC''
  \item
    Jul 31, 2004: 1st summit; Nepal and Bhutan joined. Renamed to
    current elaboration.
  \end{itemize}
\item
  \textbf{HQ:} Dhaka
\item
  \textbf{Summits:}

  \begin{itemize}
  \tightlist
  \item
    1st: Jul 31, 2004
  \item
    5th: Mar 30, 2022
  \end{itemize}
\end{itemize}
\end{frame}

\section{BENELUX}
\begin{frame}[allowframebreaks]
{BENELUX}
\protect\hypertarget{benelux}{}
\begin{itemize}
\tightlist
\item
  Belgium, Netherlands, Luxembourg Economic Cooperation
\item
  \textbf{Establishment:}

  \begin{itemize}
  \tightlist
  \item
    Benelux customs union treaty:

    \begin{itemize}
    \tightlist
    \item
      Signed on Sep 5, 1944
    \item
      Effective from Jan 1, 1948
    \end{itemize}
  \end{itemize}
\end{itemize}
\end{frame}

\section{ECO (Economic Cooperation Organisation)}
\begin{frame}[allowframebreaks]
{ECO (Economic Cooperation Organisation)}
\protect\hypertarget{eco-economic-cooperation-organisation}{}
\begin{itemize}
\tightlist
\item
  \textbf{Establishment:}

  \begin{itemize}
  \tightlist
  \item
    1964: RCD (regional coop for development) formed by Iran, Turkey,
    Pakistan
  \item
    1985: ECO, successor of RCD
  \end{itemize}
\item
  \textbf{Members:} 10
\item
  \textbf{HQ:} Tehran
\end{itemize}
\end{frame}

\section{OECD}
\begin{frame}[allowframebreaks]
{OECD}
\protect\hypertarget{oecd}{}
\begin{itemize}
\tightlist
\item
  Organisation for Economic Cooperation and Development
\item
  \textbf{Established:}

  \begin{itemize}
  \tightlist
  \item
    Apr 16, 1948: OEEC (org for european econ coop)
  \item
    1961: reformed to OECD
  \end{itemize}
\item
  \textbf{Members:} 38
\item
  \textbf{HQ:} Paris
\item
  \textbf{Sec. Gen.:} Mathias Cormann (Australia)
\end{itemize}
\end{frame}

\end{document}
